\section{Theoretical Analysis}
\label{sec:analysis}

In this section, the circuit shown in Figure~\ref{fig:rc} is analysed
theoretically, in terms of voltages and currents in each node and branches respectively.


\section{Nodal Method}




%\lipsum[1-1]


\section{Mesh Method}

By analysing the circuit in terms of the elementar meshes, we started by assuming that we have $4$ unknown quantities: $I_a, I_b, I_c, I_d$ (four meshes).
With no effort, we realized that the current in the mesh of the lower right corner is defined by the current source $I_d$.
Using the fact that the voltage-controlled current source presented in the mesh of $I_b$ only belongs to this elementar mesh and that $I_b = K_b V_b$ (formula developed in \ref{restrict1}), there is no need in writing an equation for the loop of current $I_b$.
As a result, we are left with 2 independent variables: $I_b$ and $I_c$ and the following equations:


%equations
\begin{equation}
  (R_1 + R_2 + R_4) \frac{K_b R_3 -1}{K_b R_3}  - R_3  - R_4 = -Va
  \label{mesh1}
\end{equation}

\begin{equation}
  -R_4 \frac{K_b R_3 - 1}{K_b R3} R_4 + R_6 + R_7 - K_c = 0
  \label{mesh2}
\end{equation}

During the derivation of the previous equations, the next restrition equations were used:

\begin{equation}
  I_a = \frac{K_b R_3 -1}{K_b R_3} I_b
  \label{restrict1}
\end{equation}

\begin{equation}
  V_c = K_c I_c
  \label{restrict2}
\end{equation}

By solving equations with a script of \textit{Octave}, we got following results presented in the table \ref{tab:op_octave}

Explain the currents, signals 'nd voltages

%signals-------------------->
\begin{table}[h]
  \centering
  \begin{tabular}{|l|r|}
    \hline
    {\bf Name} & {\bf Value [A or V]} \\ \hline
    I3 & -1.18476e-05 hline
I4 0.00122074 hline
  \end{tabular}
  \caption{Results of Mesh Analysis}
  \label{tab:op_octave}
\end{table}


%\lipsum[1-1]


