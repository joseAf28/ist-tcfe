\section{Theoretical Analysis}
\label{sec:analysis}

\subsection{OP AMP model}

In the theoretical part, we use the ideal model of the OP-AMP. So, we consider the input impedance equals to $\infty \Omega$, the ouput impedance
equals to $0 \Omega$, and the gain equal to $\infty$. As a consequence, the differences of voltage of the non inverting input $v_{+}$ and The
inverting input $v_{-}$ is equal to zero and we have a \textit{virtual ground} in the non inverting input.


\subsection{Frequency response}

In order to obtain, the transfer function, we use nodal analysis method to obtain the following system of equations:

\[
  \begin{bmatrix}
    \frac{1}{R_1} + \frac{1}{Z_{C1}} + \frac{1}{Z_{C2}} & -\frac{1}{Z_{C2}} \\
    -\frac{1}{Z_{C1}}                                   & -\frac{1}{R_2}    \\
  \end{bmatrix}
  \begin{bmatrix}
    V_{1} \\ V_{out}
  \end{bmatrix}
  =
  \begin{bmatrix}
    \frac{V_{in}}{R_1} \\ 0
  \end{bmatrix}
\]

\hfill



\begin{equation}
  T(\omega) = \frac{\widetilde{V_{out}}}{\widetilde{V_{in}}}
  \label{frequencyR}
\end{equation}

We use Octave to solve this system of equations numerically for a differences range of frequencies of the input signal.

In plot \ref{fig:output1}, we have the phase, obtained by taking the $arg(T(\omega))$, as function of log (f) in Hz.

Input Impedance Output Impedance

\begin{figure*}[h]
  \centering
  \begin{subfigure}{0.5\textwidth}
    \includegraphics[width=\linewidth, clip]{phase.eps}
    \label{fig:output1}
  \end{subfigure}
  \begin{subfigure}{0.4\textwidth}
    \includegraphics[width=\linewidth, clip]{phase.pdf}
    \label{fig:output2}
  \end{subfigure}
  \caption{\small  )}
  \label{output_deviation}
\end{figure*}


In plot \ref{fig:argTOct} we have the magnitude, calculated from $20*log_{10}(abs(T(\omega)))$ as function of log(f) .

\begin{figure*}[h]
  \centering
  \begin{subfigure}{0.5\textwidth}
    \includegraphics[width=\linewidth, clip]{gain.eps}
    \label{fig:argTOct}
  \end{subfigure}
  \begin{subfigure}{0.4\textwidth}
    \includegraphics[width=\linewidth, clip]{gain.pdf}
    \label{fig:argTSpi}
  \end{subfigure}
  \caption{\small }
  \label{}
\end{figure*}

\subsection{Input and Output Impedance and Central Frequency}

To evaluate the frequency of the input signal whose gain is maximized, we applied the \textit{max} function of Octave to the array in which
we have the valus of the magnitude of the gain. The value we get is presented in the table \ref*{tab:TheoreticalResults}.

In order to calculate the input impedance of the circuit in central frequency,
we use the following formula with a frequency of the input signal we get before:

\begin{equation}
  Z_{input} = \frac{V_[in]}{\frac{V_{in} - V_1}{R_1}} \frac{1}{1000} k \Omega
\end{equation}

Due to the teoretical model used for the OP-AMP, we have an $Z_{output} = 0$.


\begin{table}[h]
  % \parbox{.45\linewidth}{
  {
    \centering
    \begin{tabular}{|c|c|c|}
      \hline
      $Variable$ & Theory (octave) & Simulation (ngspice) \\ \hline 
$Z_{input} [k \Omega]$ & 1.312927e-01 + -1.634127e-01i & 1.352850e-01 + -1.156956e-01i \\ \hline
$Z_{output} [k \Omega]$ & 0.000000e+00 & -2.651410e-03 + -2.173550e-02i \\ \hline
Max $Gain [dB]$ & 1.835867e+01 & 1.827262e+01 \\ \hline
Central frequency [Hz] & 9.660509e+02 & 1.000150e+03 \\ \hline

    \end{tabular}
    \label{tab:TheoreticalResults}
    \caption{ }
  }
\end{table}

