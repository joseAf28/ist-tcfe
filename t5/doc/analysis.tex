\section{Theoretical Analysis}
\label{sec:analysis}

In this circuit, we have a multiple negative feedback applied via the equivalent resistor R2 and the
and the equivalent capacitor C2.

Moreover, the capcitor C2 blocks the input signals with low frequency and the capacitor C1
blocks the input signal with high frequencies.

\subsection{OP-AMP model}

In the theoretical part, we used the ideal model of the OP-AMP. So, we considered the input impedance equal to $\infty \Omega$, the ouput impedance
equal to $0 \Omega$, and the gain equal to $\infty$. As a consequence, the difference of voltage between the non inverting input $v_{+}$ and the
inverting input $v_{-}$ is equal to zero and we have a \textit{virtual ground} in the non inverting input.


\subsection{Frequency response}

In order to obtain the transfer function, we used nodal analysis to obtain the following system of equations:

\[
  \begin{bmatrix}
    \frac{1}{R_1} + \frac{1}{Z_{C1}} + \frac{1}{Z_{C2}} & -\frac{1}{Z_{C2}} \\
    -\frac{1}{Z_{C1}}                                   & -\frac{1}{R_2}    \\
  \end{bmatrix}
  \begin{bmatrix}
    V_{1} \\ V_{out}
  \end{bmatrix}
  \begin{bmatrix}
    \frac{V_{in}}{R_1} \\ 0
  \end{bmatrix}
\]

\hfill

where
\begin{equation}
  C1 = \frac{1}{\frac{3}{10^{-6} F}} \hspace{1cm} C2 = \frac{1}{\frac{3}{220*10^{-9} F}}
\end{equation}

and

\begin{equation}
  R1 = \frac{1}{\frac{3}{1 k \Omega} + \frac{3}{100 k \Omega}} \hspace{1cm} R2 = \frac{1}{\frac{3}{10 k\Omega}}
\end{equation}


To calculate the transfer function, we use the Octave script to calculate the next formula for the range of frequencies:

\begin{equation}
  T(\omega) = \frac{\widetilde{V_{out}}}{\widetilde{V_{in}}}
  \label{frequencyR}
\end{equation}

We used Octave to solve this system of equations numerically for a range of frequencies of the input signal. The results will be shown in section \ref{sec:side}:
in plot \ref{fig:argTOct}, in the left, we have the magnitude, calculated from $20*log_{10}(abs(T(j\omega)))$ as function of log(f).
In plot \ref{fig:output1}, in the left, we have the phase, obtained by taking the $arg(T(\omega))$, as function of log (f) in Hz.

\subsection{Input and Output Impedance and Central Frequency}

To evaluate the frequency of the input signal whose gain is maximized, we applied the \textit{max} function of Octave to the array in which
we have the values of the magnitude of the gain. The value we get is presented in the table \ref*{tab:TheoreticalResults}.

In order to calculate the input impedance of the circuit in the central frequency,
we used the following formula with the frequency of the input signal we got before:

\begin{equation}
  Z_{input} = \frac{V_{in}}{\frac{V_{in} - V_1}{R_1}} \frac{1}{1000} k \Omega
\end{equation}

Due to the teoretical model used for the OP-AMP, we have an $Z_{output} = 0$.

