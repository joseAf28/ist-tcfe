\section{Theoretical Analysis}
\label{sec:analysis}

In this section, the circuit $RC$ shown in Figure~\ref{fig:circuit} is analysed
theoretically.


\subsection{Nodal Method}

We start by doing the nodal analysis of the circuit in order to find out the potentials and currents for all nodes and branches respectively for $t < 0$????,
where there was a steady state for the circuit and $V_s = 5.18382634375 V$ . By noticing that the node 0 is used as the $0 V$ reference node, we get the following system of equations:



%b = [Id; -Va/R1; Va/R1; Id]

\[
  \begin{bmatrix}
    - \frac{1}{R_1} & \frac{1}{R_1} + \frac{1}{R_2} + \frac{1}{R_3} & -\frac{1}{R_2} & -\frac{1}{R_3}                                & 0              & 0                              & 0             \\
    0               & K_b + \frac{1}{R_2}                           & -\frac{1}{R_3} & - K_b                                         & 0              & 0                              & 0             \\
    0               & 0                                             & 0              & 0                                             & 0              & -\frac{1}{R_6} - \frac{1}{R_7} & \frac{1}{R_7} \\
    0               & K_b                                           & 0              & -K_b - \frac{1}{R_5}                          & \frac{1}{R_5}  & 0                              & 0             \\
    0               & -\frac{1}{R_3}                                & 0              & \frac{1}{R_3} + \frac{1}{R_4} + \frac{1}{R_5} & -\frac{1}{R_5} & -\frac{1}{R_7}                 & \frac{1}{R_7} \\
    0               & 0                                             & 0              & 1                                             & 0              & \frac{K_d}{R_6}                & -1            \\
    1               & 0                                             & 0              & 0                                             & 0              & 0                              & 0             \\
  \end{bmatrix}
  \begin{bmatrix}
    V_1 \\ V_2 \\ V_3 \\ V_5 \\ V_6 \\ V_7 \\ V_8
  \end{bmatrix}
  =
  \begin{bmatrix}
    0 \\ 0 \\ 0 \\ 0 \\ 0  \\ 0 \\ V_s
  \end{bmatrix}
\]

\hfill


An Octave script was prepared to solve this system numerically. The results are shown in Table \ref{tab:op_tabNodal2}.


%meter as tabelas juntas
\begin{table}[b]
  \centering
  \begin{tabular}{|c|c|}
    \hline
    {\bf Name} & {\bf Value [A or V]} \\ \hline
    \input{op_nodal1_tab}
  \end{tabular}
  \caption{Results of Nodal Analysis for $t < 0$. A variable that begins  with \textit{I} names a \textit{current} in \textit{Ampere}; the ones that start with \textit{V} name a \textit{voltage} in \textit{Volt} }
  \label{tab:op_tabNodal2}
\end{table}



\subsection{Determination of the Equivalemt Resistance seen by the Capacitor}

In order to determine the equivalent resistance, we start by switching off the independent source $V_s$ of the circuit. Due to the fact we can't switch off the dependent sources
as done with the independent source, we replace the capacitor with an independent voltage source with a tension of $V6-V8 = V_{68} = 8.768729 V$ (result of the previous section) and we use the nodal analysis, as used before,
to determine the currents and voltages in every branch and node respectively. As a consequence, the potential in node $1$ and node $0$ are $V_0 = V_1 = 0 V$

We get the following system of equations:

\[
  \begin{bmatrix}
    -\frac{1}{R_1}                                & 0              & -\frac{1}{R_4} & 0 & -\frac{1}{R_6}                & 0                    \\
    \frac{1}{R_1} + \frac{1}{R_2} + \frac{1}{R_3} & -\frac{1}{R_2} & -\frac{1}{R_3} & 0 & 0                             & 0                    \\
    -K_b - \frac{1}{R_2}                          & \frac{1}{R_2}  & K_b            & 0 & 0                             & 0                    \\
    0                                             & 0              & 0              & 0 & \frac{1}{K_d} + \frac{1}{R_7} & -\frac{1}{R_7}       \\
    0                                             & 0              & 0              & 1 & 0                             & \frac{K_d}{R_6} & -1 \\
    0                                             & 0              & 0              & 0 & 1                             & 0               & -1 \\
  \end{bmatrix}
  \begin{bmatrix}
    V_2 \\ V_3 \\ V_5 \\ V_6 \\ V_7 \\ V_8
  \end{bmatrix}
  =
  \begin{bmatrix}
    0 \\ 0 \\ 0 \\ 0  \\ 0 \\ V_{68}
  \end{bmatrix}
\]

\hfill

\begin{table}[b]
  \centering
  \begin{tabular}{|l|r|}
    \hline
    {\bf Name} & {\bf Value [A or V]} \\ \hline
    \input{op_nodal5_tab}
  \end{tabular}
  \caption{Results of Nodal Analysis of the circuit used to determine the equivalent resistance seen by the capacitor. A variable that begins  with \textit{I} names a \textit{current} in \textit{Ampere}; the ones that start with \textit{V} name a \textit{voltage} in \textit{Volt} }
  \label{tab:op_tabReq}
\end{table}


By solving these equations with a script of \textit{Octave}, we got the following results presented in the table \ref{tab:op__tabReq}.


By looking at first glance, the results do not seem right. So we use an alternative method to find the equivalent resistance: we replace the capacitor with the an indepent voltage source of $1 A$, in order to do the calculation of the equivalent resistance easier,
and we run mesh analysis in the circuit with the elementar currents shown in \ref{fig:circuit}. The results are shown in the following system of equations: REFF

\[
  \begin{bmatrix}
    0 & 1-K_b R3 & K_b R_3         & 0                     \\
    0 & 0        & -R_4            & R_6 + R_7 - K_d + R_4 \\
    0 & 0        & R_1 + R_3 + R_4 & -R_4                  \\
    1 & 0        & 0               & 0                     \\
  \end{bmatrix}
  \begin{bmatrix}
    I_b \\ I_a \\ I_d \\ I_{c'}
  \end{bmatrix}
  =
  \begin{bmatrix}
    0 \\ 0 \\ 0 \\ 1
  \end{bmatrix}
\]

\hfill


As can be senn easily, the first three equations are linearly independent of each other and the last equations is disconnect from the other ones. So, $I_2 = I_3 = I_4 = 0$. As result, from the last equation, we know that
the equivalent resistance is equal to $R_5$



%\lipsum[1-1]

\subsection{Natural solution of v6(t)}

By noticing that at $t = 0$ the voltage source changes its behaviour and that the potential difference across the capacitor must changes in a continuous way (the electric potential is a continuous funcion, even  if the electric field is discontinuous)
we get the following results:

The time constant is given by $\tau = R_{eq}C$.
From the general natural solution of an RC circuit:

\begin{equation}
  v6 (t) = v_6(\infty) + (v_6(0) - v_6(\infty)) e^{-\frac{t}{R_{eq}C}}
\end{equation}


By noticing from now on the independent source is switched off and that as $t \rightarrow \infty$, the energy of the circuit is dissipated by the resistors, the $V_c(\infty) = 0$
At $t = 0$, we get that $V_6 = 8.768729V$, as seen in the previous section.

So, we have as natural solution of the system:

\begin{equation}
  v6 (t) = v_6(0)e^{-\frac{t}{R_{eq}C}}
  \label{eqNaturalSol}
\end{equation}

By running this function in \textit{Octave}, we get the following plot \ref{eqNaturalSol}

\begin{figure}[h] \centering
  \includegraphics[width=0.6\linewidth]{v6n.eps}
  \caption{Natural solution of $v_6(t)$ }
  \label{fig:naturalSolution}
\end{figure}


\subsection{Forced solution of v6(t)}

In order to find the forced solution, we start by transforming the AC voltage of the source to a phasor.

\begin{equation}
  V_s = sin(2\pi f t) = \Re (e^{j(2\pi f t - \frac{\pi}{2})}) \implies \tilde{V_s} = e^{-j \frac{\pi}{2}} = -j
\end{equation}

As a result of using phasors, we have to use the impedance of each element, quantity that relates the phasor voltage and the current phasor: $Z = \frac{\tilde{V}}{\tilde{I}} $

It is known that impedance of a resistor $Z_r = R_c$ (it is just a real number). The capacitor impedance is given by the following impedance $Z_c = \frac{1}{j\omega C}$.
The fact the it is just an imaginary number, it tells that the capacitor does not dissipate energy out of the circuit.

By using the nodal analysis in a similar way as used before, we get the following system of equations:

\[
  \begin{bmatrix}
    - \frac{1}{R_1} & \frac{1}{R_1} + \frac{1}{R_2} + \frac{1}{R_3} & -\frac{1}{R_2} & -\frac{1}{R_3}                                & 0                          & 0                              & 0                          \\
    0               & K_b + \frac{1}{R_2}                           & -\frac{1}{R_3} & - K_b                                         & 0                          & 0                              & 0                          \\
    0               & 0                                             & 0              & 0                                             & 0                          & -\frac{1}{R_6} - \frac{1}{R_7} & \frac{1}{R_7}              \\
    0               & K_b                                           & 0              & -K_b - \frac{1}{R_5}                          & \frac{1}{R_5} + j\omega C  & 0                              & -j\omega C                 \\
    0               & -\frac{1}{R_3}                                & 0              & \frac{1}{R_3} + \frac{1}{R_4} + \frac{1}{R_5} & -\frac{1}{R_5} - j\omega C & -\frac{1}{R_7}                 & \frac{1}{R_7} + j \omega C \\
    0               & 0                                             & 0              & 1                                             & 0                          & \frac{K_d}{R_6}                & -1                         \\
    1               & 0                                             & 0              & 0                                             & 0                          & 0                              & 0                          \\
  \end{bmatrix}
  \begin{bmatrix}
    V_1 \\ V_2 \\ V_3 \\ V_5 \\ V_6 \\ V_7 \\ V_8
  \end{bmatrix}
  =
  \begin{bmatrix}
    0 \\ 0 \\ 0 \\ 0 \\ 0  \\ 0 \\ -1
  \end{bmatrix}
\]

\hfill


The results of the complex amplitudes are shwon in the following table:

\begin{table}[b]
  \centering
  \begin{tabular}{|l|c|}
    \hline
    {\bf Name} & {\bf Phasor} \\ \hline
    \input{op_nodal2_tab}
  \end{tabular}
  \caption{Complex amplitudes in the nodes of the circuit}
  \label{tab:op_tabNodal}
\end{table}


From the previous calculations, we have as forced solution in $v_6$:

\begin{equation}
  v_{6f} = \Re (\tilde{V_6} e^{j 2\pi f t}) = V_{6f} cos(2 \pi f t + \phi)
  \label{forcedSolution}
\end{equation}


\subsection{General Solution of v6(t)}

By doing a linear combination of the natural and the forced solutions found in the previous sections, we have as general solution the following function:

\begin{equation}
  v_6(t) = V _6(0)e^{-\frac{t}{R_{eq}C}} + V_{6f} cos(2 \pi f t + \phi)
  \label{finalSolution}
\end{equation}

By using \textit{Octave} to plot this function, we get the plot \ref{fig:generalFinal}:

\begin{figure}[h] \centering
  \includegraphics[width=0.6\linewidth]{v6_vs.eps}
  \caption{Genereal solution of v6(t)}
  \label{fig:generalFinal}
\end{figure}


As can be seen , the plot of $v_6(t)$ have a discontinuity at $t = 0$.
This occurs to satisfy the restriction imposed by the conservation of the energy  of the electric field across the capacitor(seen here in the form of a potential difference).


\subsection{Frequency response}

In order to evaluate the frequency response of the circuit, we have to find a way of evaluating the following equation:

\begin{equation}
  T(\omega) = \frac{\widetilde{V_{out}}}{\widetilde{V_{in}}}
  \label{frequencyR}
\end{equation}

where in function \ref{frequencyR} we have $\widetilde{V_{out}} = \tilde{V_6}$ and $\widetilde{V_{out}} = \tilde{V_c}$ as been asked.

Because of the difficulty in solving the system of equation found in the section of \textit{Forced Solution} to arrange the function \ref{frequencyR} explicitly, we find out the solution numerically using Octave.

In plot \ref{fig:argT}, we have the phase, obtained by taking the $arg(T(\omega))$, as function of $log(\omega)$ for the range of $0.1Hz$ to $1MHz$.

\begin{figure}[h] \centering
  \includegraphics[width=0.6\linewidth]{argT.eps}
  \caption{Genereal solution of v6(t)}
  \label{fig:argT}
\end{figure}

In plot \ref{fig:absT} we have the magnitude, calculated from $20*log_{10}(abs(T(\omega)))$ as function of $log(\omega)$ for the same range of frequencies as given before.

\begin{figure}[h] \centering
  \includegraphics[width=0.6\linewidth]{absT.eps}
  \caption{Genereal solution of v6(t)}
  \label{fig:absT}
\end{figure}


As we can see, the phase of phase and the magnitude of $T_{Vs}$ is constant as imposed by the source.
From the plots of $T_{Vc}$, we see that the response of the capacitor for frequencies smaller than its natural frequency $f_n = \frac{1}{R_{eq}C} = 318.674 Hz$, the output voltage is almost equal to input voltage and that the signals have the same phase.
For frequencies much bigger than his natural frequency, the $V_{out} \rightarrow 0$, as seen  in \ref{fig:absT} and the signal is $-180$ degrees ou of phase.
For frequencies in a near range of the natural frequency, we have an abrupt decline in $V_{out}$ and phase.

In regard to $T_{V6}$, we have