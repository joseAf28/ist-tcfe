\section{Simulation Analysis}
\label{sec:simulation}

In this section the circuit in figure \ref{fig:circuit} is simulated
using the software \textit{Ngspice}.

\subsection{Operating Point Analysis}

\subsubsection{t<0}


We started by simulating the static circuit for  $time (t) < 0$ and 
got the voltages and currents in each node and branch of the circuit.
Table~\ref{tab:op_tab} shows the simulated operating 
point results for the circuit under analysis.
Note that, because at $t<0$ the voltage is constant, the current through
the capacitor is zero, i.e., it can be replaced by an open circuit.

\hfill

\begin{table}[ht]
  \centering
  \begin{tabular}{|l|r|}
    \hline
    {\bf Name} & {\bf Value [A or V]} \\ \hline
    @gib[i] & -2.65517e-04\\ \hline
@id[current] & 1.025904e-03\\ \hline
@r1[i] & -2.53670e-04\\ \hline
@r2[i] & -2.65517e-04\\ \hline
@r3[i] & -1.18476e-05\\ \hline
@r4[i] & 1.220737e-03\\ \hline
@r5[i] & -1.29142e-03\\ \hline
@r6[i] & 9.670671e-04\\ \hline
@r7[i] & 9.670671e-04\\ \hline
v(1) & -7.95657e+00\\ \hline
v(2) & 3.950179e+00\\ \hline
v(3) & -3.64391e-02\\ \hline
v(4) & -4.95541e+00\\ \hline
v(5) & -6.94792e+00\\ \hline
v(6) & -5.78231e-01\\ \hline
v(7) & 2.284139e-01\\ \hline
v(8) & -4.95541e+00\\ \hline

  \end{tabular}
  \caption{Operating point for $t<0$. A variable preceded by @ is 
  of type {\em current} and expressed in Ampere; other variables 
  are of type {\it voltage} and expressed in Volt.}
  \label{tab:op_tab}
\end{table}

\subsubsection{*TITULO*}

Following this we replaced $V_s$ with a short circuit and substituted
the capacitor with a voltage source $Vx = V_8-V_6$, $V_8$ and $V_6$
being the potencial at nodes 8 and 6 previously calculated.
From this we obtained the values in table \ref{tab:opb_tab}. 

As explained in section (*METER LINK DA SECÇAO*), we had to substitute 
the capacitor with a voltage source because we cannot replace the 
independent source with short circuits or open circuits, as we did 
with $V_s$ and simply calculate the equivalent resistance using
simple resistance sum formula.

By comparing the results exposed in table \ref{tab:opb_tab} with the 
theoretical results in table (\ref{tab:op_nodal5_tab} <-TABELA TEORICA), we 
find that the results agree up to the last decimal place represented 
by \textit{Ngspice} (5 decimal places). This agreement can be explained 
with the fact that there are only linear and time independent components 
in the circuit we are considering.


\subsection{Transient Analysis}

In this section we analysed the response of the circuit for several
conditions.

\subsubsection{Natural Response}

We started with the natural response of the circuit, that is, 
with $V_s=0$, to a boundary condition ($V_6$ and $V_8$ as determined 
in the previous section).
Using ngspice's transient analysis we were able to simulate the response. 
The value of $V_6$ is plotted in \ref{} **INSERIR FIGURA**.

\subsubsection{Total Responce}

Afterwards, we simulated the total response of the full circuit (natural 
+ forced) by assuming $V_s = sin(\omega t)$ (as explained in figure \ref{fig:circuit}), 
for a frequency of 1kHz.
The voltage in node 6 ($V_6$) and the input voltage ($V_s$) are plotted 
in figure \ref{}

**inserir figura

\subsubsection{Frequency }

Finally, with the goal of observing the low-pass filter nature 
of the capacitor, we simulated the circuit for various frequencies of 
$V_s$ (0.1Hz to 1MHz) and plotted the change in the phase (\ref{}) 
and in the amplitude (\ref{}) of $V_s$ and $V_6$. Note that the 
amplitude is expressed in dB's, justifying the negative values.




%\lipsum[1-1]




