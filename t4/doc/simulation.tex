\section{Simulation Analysis}
\label{sec:simulation}


\begin{figure*}[h]
    \centering
    \begin{subfigure}{0.23\textwidth}
        \includegraphics[width=\linewidth, clip]{vo1.pdf}
        \label{fig:output1}
    \end{subfigure}
    \begin{subfigure}{0.23\textwidth}
        \includegraphics[width=\linewidth, clip]{vo1f.pdf}
        \label{fig:output2}
    \end{subfigure}
    \caption{\small $V_Out - 12$ - measure of the output DC deviation + AC component (left - simulation; right - theoretical )}
    \label{output_deviation}
\end{figure*}

\begin{figure*}[h]
    \centering
    \begin{subfigure}{0.23\textwidth}
        \includegraphics[width=\linewidth, clip]{vo2f.pdf}
        \label{fig:output1}
    \end{subfigure}
    \begin{subfigure}{0.23\textwidth}
        \includegraphics[width=\linewidth, clip]{gainACtotal.eps}
        \label{fig:output2}
    \end{subfigure}
    \caption{\small $V_Out - 12$ - measure of the output DC deviation + AC component (left - simulation; right - theoretical )}
    \label{output_deviation}
\end{figure*}

% \begin{figure}
%     \centering
%     \begin{subfigure}{.5\textwidth}
%         \centering
%         \includegraphics[width=.4\linewidth]{vo1.pdf}
%         \caption{A subfigure}
%         \label{fig:sub1}
%     \end{subfigure}%
%     \begin{subfigure}{.5\textwidth}
%         \centering
%         \includegraphics[width=.4\linewidth]{gainACtotal.eps}
%         \caption{A subfigure}
%         \label{fig:sub2}
%     \end{subfigure}
%     \caption{A figure with two subfigures}
%     \label{fig:test}
% \end{figure}


\begin{table}
    \parbox{.45\linewidth}{
        \centering
        \begin{tabular}{|c|c|}
            \hline
            {\bf Name} & {\bf Value [A or V]} \\ \hline
            \input{GainStage_OP.tex}
        \end{tabular}
        \label{tab:GainStage_OP}
        \caption{Results of the DC analysis applied to the Gain Stage circuit}
    }
    \hfill
    \parbox{.45\linewidth}{
        \centering
        \begin{tabular}{|c|c|}
            {\bf Name} & {\bf Value [A or V]} \\ \hline
            @gib[i] & -2.65517e-04\\ \hline
@id[current] & 1.025904e-03\\ \hline
@r1[i] & -2.53670e-04\\ \hline
@r2[i] & -2.65517e-04\\ \hline
@r3[i] & -1.18476e-05\\ \hline
@r4[i] & 1.220737e-03\\ \hline
@r5[i] & -1.29142e-03\\ \hline
@r6[i] & 9.670671e-04\\ \hline
@r7[i] & 9.670671e-04\\ \hline
v(1) & -7.95657e+00\\ \hline
v(2) & 3.950179e+00\\ \hline
v(3) & -3.64391e-02\\ \hline
v(4) & -4.95541e+00\\ \hline
v(5) & -6.94792e+00\\ \hline
v(6) & -5.78231e-01\\ \hline
v(7) & 2.284139e-01\\ \hline
v(8) & -4.95541e+00\\ \hline

            \hline
        \end{tabular}
        \label{tab:Spice_OP}
        \caption{Results of the OP analysis applied to the whole circuit by using the Ngspice}
    }
    \hfill
    \parbox{.45\linewidth}{
        \centering
        \begin{tabular}{|c|c|}
            {\bf Name} & {\bf Value [A or V]} \\ \hline
            VI2 & 3.057109e+00\\ \hline
IC2 & 8.206785e-02\\ \hline
IE2 & 8.242891e-02\\ \hline
VO2 & 3.757109e+00\\ \hline

            \hline
        \end{tabular}
        \label{tab:OutputStage_OP}
        \caption{Results of the DC analysis applied to the Output Stage circuit }
    }
\end{table}

Compare the Operating point both models......



\begin{table}
    \parbox{.45\linewidth}{
        \centering
        \begin{tabular}{|c|c|}
            \hline
            {\bf Name} & {\bf Impedance or Gain in dB} \\ \hline
            gm1 & 3.577156e-01\\ \hline
$r pi 1$ & 4.995588e+02\\ \hline
r01 & 7.793900e+03\\ \hline
AV1 & -2.627909e+02\\ \hline
$AV1_{DB}$ & 4.839221e+01\\ \hline
ZI1& 4.844336e+02\\ \hline
ZO1 & 8.862848e+02\\ \hline

        \end{tabular}
        \label{tab:GainStage_AC}
        \caption{Results of the Incremental Analysis applied to the Gain Stage in the theoretical part}
    }
    \hfill
    \parbox{.45\linewidth}{
        \centering
        \begin{tabular}{|c|c|}
            {\bf Name} & {\bf Impedance or Gain in dB} \\ \hline
            gm2 & 3.282714e+00\\ \hline
$r pi 2$ & 6.924149e+01\\ \hline
r02 & 4.532835e+02\\ \hline
AV2 & 9.919476e-01\\ \hline
$AV_{DB}$ & -7.022544e-02\\ \hline
ZI2 & 8.598855e+03\\ \hline
ZO2 & 3.021730e-01\\ \hline

            \hline
        \end{tabular}
        \label{tab:OutputStage_AC}
        \caption{Results of the Incremental Analysis applied to the Output Stage in the theoretical part }
    }
    \hfill
    \parbox{.45\linewidth}{
        \centering
        \begin{tabular}{|c|c|}
            {\bf Name} & {\bf Value [A or V]} \\ \hline
            AV & -2.500181e+02\\ \hline
$AV_{DB}$ & 4.795943e+01\\ \hline
ZI & 4.844336e+02\\ \hline
ZO & 3.981969e+00\\ \hline

            \hline
        \end{tabular}
        \label{tab:Final}
        \caption{Impedance of input and output of the Amplifier and total gain of the circuit}
    }
\end{table}



