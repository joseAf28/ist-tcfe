\section{Theoretical Analysis}
\label{sec:analysis}

In this section, we use two different simplified models that are able to describe the behaviour of the diode
in order to study the AC/DC converter developed in the Simulation Analysis


\section{Envelope Detector}

In order to maximize the usage of diodes, we create a full wave rectifier circuit. We use the Diode Model with ideal diode + voltage source.
So we have that each diode works according to the following equation:


\[
  \left\{
  \begin{array}{ll}
    i = 0, if v < V_{ON}   \\
    v = V_{ON}, if  i \geq \\
  \end{array}
  \right.
\]

Due to the fact we have a full wave rectifier, we have two diodes in series either the voltage source be positive or negative.
Moreover, the output voltage of this bridge, $V_o$:

\[
  \left\{
  \begin{array}{ll}
    V_o = 0,  if  |V_s| < 2V_{ON}               \\
    v = |V_s| - 2 V_{ON}, if |V_s| \geq 2V_{ON} \\
  \end{array}
  \right.
\] where $V_s$ is the voltage of the source

In order to smooth the signal of the output voltage, we add a capacitor and a resistor in parallel
WITH TERMINALS????????.

So we have that initially the potential difference between the terminal of the capacitor is $V_o$. From the time instant $t_{OFF}$
the capacitor starts to discharge through the resistor because is voltage is bigger than $V_o$ at this point, which smooths the output voltage.
During this period, the diodes are inversely polarized and they do not enable to pass charge.

So during this interval of time, we have:

\begin{equation}
  i_R = - i_C \Leftrightarrow \frac{v_s}{R} = -C\frac{dv_s}{dt} \Leftrightarrow \frac{A}{R}\cos(\omega t_{OFF}) = C A \omega \sin(\omega t_{OFF})
\end{equation}

\begin{equation}
  t_{OFF} = \frac{1}{\omega}\arctan\frac{1}{\omega RC}
\end{equation}

So from $t \geq t_{OFF}$, we have:

\begin{equation}
  v_o(t) = A\cos(\omega t_{OFF}) e^{-\frac{t-t_{OFF}}{RC}}
\end{equation}

At $t = t_{ON}$, the diodes goes ON again:

\begin{equation}
  v_o(t_{ON}) = A\cos(\omega t_{OFF}) e^{-\frac{t_{ON}-t_{OFF}}{RC}}
\end{equation}

We solve this equation with \textit{Octave}.


\section{Voltage Regulator}

By using the non linear behaviour of the diodes, we create a voltage regulator that uses 18 diodes.
We use the diode model with an ideal diode+ voltage source + resistor. we consider the resistance of incremental analysis of the diode
$r_d = 41.460 Ohm$ and $V_{ON} = 0.66787 V$.

Using Incremental analysis: $v_= = V_o + v_o$, where $V_o$ is the operating point and $v_o$ incremntal voltage.

We use the following equation:

\begin{equation}

\end{equation}
%\lipsum[1-1]


