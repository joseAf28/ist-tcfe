\section{Simulation Analysis}
\label{sec:simulation}

\subsection{Operating Point Analysis}

By creating a script of \textit{Ngspice}, we got the voltages and currents in each node and branch of
the circuit \ref{fig:circuit} respectively. Table~\ref{tab:op_tab} shows the simulated operating point results for the circuit
under analysis.
\hfill
As can be seen, by comparing the results of the tables \ref{tab:op_nodal_tab} (nodal analysis) and \ref{tab:op_mesh_tab} (mesh analysis)
with the table \ref{tab:op_tab}, we got the same results.

\begin{table}[ht]
  \centering
  \begin{tabular}{|l|r|}
    \hline
    {\bf Name} & {\bf Value [A or V]} \\ \hline
    @gib[i] & -2.65517e-04\\ \hline
@id[current] & 1.025904e-03\\ \hline
@r1[i] & -2.53670e-04\\ \hline
@r2[i] & -2.65517e-04\\ \hline
@r3[i] & -1.18476e-05\\ \hline
@r4[i] & 1.220737e-03\\ \hline
@r5[i] & -1.29142e-03\\ \hline
@r6[i] & 9.670671e-04\\ \hline
@r7[i] & 9.670671e-04\\ \hline
v(1) & -7.95657e+00\\ \hline
v(2) & 3.950179e+00\\ \hline
v(3) & -3.64391e-02\\ \hline
v(4) & -4.95541e+00\\ \hline
v(5) & -6.94792e+00\\ \hline
v(6) & -5.78231e-01\\ \hline
v(7) & 2.284139e-01\\ \hline
v(8) & -4.95541e+00\\ \hline

  \end{tabular}
  \caption{Operating point. A variable preceded by @ is of type {\em current}
    and expressed in Ampere; other variables are of type {\it voltage} and expressed in
    Volt. Note that r8 was a fictitous resistance used so that Ngspice would show the value of the current $I_{Vc}$}
  \label{tab:op_tab}
\end{table}


%\lipsum[1-1]




